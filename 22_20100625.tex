\section{B-Splines}
B = Basis (nicht Bézier)
\begin{center}
 % 22.1
\end{center}
Der B-Spline ist bei Ordnung 1 das Kontrollpolygon selbst.\\[1em]
Knoten $u_1 < u_2 < u_3 < ... < u_n \in \mathbb{R}$
\[
 X(t) = \frac{u_{i+1}-t}{u_{i+1}-u_i} P_{i+1} + \frac{t-u_i}{u_{i+1}-u_i} P_i \qquad u_i \le t \le u_{i+1}
\]
Notation:
\begin{center}
 % 22.2
\end{center}
$P$ interpoliert zwischen $A$ und $B$, wenn $t$ von $r$ bis $s$ variiert
\[P(t) = \frac{s-t}{s-r} A + \frac{t-r}{s-r} B \qquad r \le t \le s\]
Wie hängt $X$ von den Punkten ab?
\[X(t) = \sum\limits_{i=0}^{n-1} \rnode{C}{C^2_i(t)} \cdot P_i\]
\begin{center}
\framebox[3cm]{\rnode{Basis}{Basisfunktionen}} \ncangle[angleA=90,angleB=-90]{->}{Basis}{C} (stückweise lineare Funktionen von $t$ hängen von
	$u_1, u_2, ...$ ab) \\[1em]
 % 22.3
\end{center}
\begin{center}
 % 22.4
\end{center}
Die Kurve existiert im Intervall $t \in [u_2, u_n]$
\begin{itemize}
 \item $n$ Punkte $P_0, P_1, ..., P_{n-1} \equiv P_0^3, ..., P_{n-1}^3$
 \item $n + 1$ Knoten $u_1 < u_2 < ... < u_{n+1}$
\end{itemize}
\paragraph*{Beispiel} $n = 5$
 \[
	\begin{array}{cccccc}
	 u_1 & u_2 & u_3 & u_4 & u_5 & u_6 \\
	 -1  & 0   & 1   & 2   & 3   & 4
	\end{array}
 \]
\begin{center}
 % 22.5\\[1em]
 % 22.6
\end{center}

\paragraph*{Basisfunktionen}
\begin{center}
 % 22.7
\end{center}

\paragraph*{kubische Splines}
\begin{center}
 % 22.8
\end{center}

\subsection{B-Splines vom Grad $\boldsymbol{d}$}
\begin{itemize}
 \item $n+d-1$ Knoten $u_1 < u_2 < ... < u_{n+d-1}$
 \item $n$ Kontrollpunkte $P_0, ..., P_{n-1}$. Diese werden als $P_i^{\boldsymbol{d+1}}$ deklariert
 \item die B-Spline-Kurve existiert für $t \in [u_d, u_n]$
	\begin{center}
	 % 22.9
	\end{center}
\end{itemize}
\[\boxed{P_i^k(t) = \frac{u_{d+k}-t}{u_{i+k}-u_i} \cdot P_{i-1}^{k+1} + \frac{t-u_i}{u_{i+k}-u_i} \cdot P_i^{k+1}},
	\qquad u_i \le t \le u_{i+k}\]
Die Punkte $P^1_i(t)$ bilden die Spline-Kurve für $t \in [u_i, u_{i+1}]$
\paragraph*{allgemeine Formel}
\[X(t) = \sum\limits_{i=0}^{n-1} C_i^k(t) P_i^k\]
Diese Formel gilt für alle $k = 1, 2, ..., d+1$. Für $k = d+1$ sind die Punkte $P_i^k$ die Kontrollpunkte und
$C_i^k(t)$ sind die Basisfunktionen

\subsection{Basisfunktion}
\begin{center}
 % 22.9
\end{center}
\[\boxed{C_i^1(t) = \begin{cases}
           1, & u_i \le t \le u_{i+1}\\
           0, & \text{sonst}
          \end{cases}}
\]
$k = 1$:
\[X(t) = \sum\limits_{i=0}^{n-1} C^1_i(t) P^1_i(t) \quad \checkmark\]
\begin{align*}
 X(t) &= \sum\limits_{i=0}^{n-1} C_i^k(t) \cdot P_i^k\\
	&= \sum\limits_{i=0}^{n-1} C_{i+1}^k(t) \cdot \frac{u_{i+k+1} - t}{u_{i+k+1} - u_{i+1}} P_{i}^{k+1}
		+ \sum\limits_{i=0}^{n-1} C_{i}^k(t) \cdot \frac{t-u_{i}}{u_{i+k} - u_i} P_{i}^{k+1}\\
	&= \sum\limits_{i=0}^{n-1} \underbrace{\left(C_{i+1}^k(t) \cdot \frac{u_{i+k+1} - t}{u_{i+k+1} - u_{i+1}}
		+ C_{i}^k(t) \cdot \frac{t-u_{i}}{u_{i+k} - u_i} \right)}_{C_i^{k+1}(t)} P_i^k+1
\end{align*}
Rekursionsformel für die Basisfunktionen $k > 1$ ($k = 1$ s. oben)
\[\boxed{C_i^k(t) = \frac{t-u_{i}}{u_{i+k-1} - u_i} C_i^{k-1}(t) + \frac{u_{i+k}-t}{u_{i+k} - u_{i+1}} C_{i+1}^{k-1}(t)}\]
$C_i^d+1(t)$ sind die Basisfunktionen for B-Splines der Ordnung $d$.

\subsection{Wahl der Knotenpunkte}
$d = 3$, kubische Splines
\begin{align*}
 u_1, u_2, u_3, ... &= 0\quad1\quad2\quad3\quad4\quad... & (\text{\footnotesize uniforme, gleichförmige B-Splines})\\
		& -2\quad-1\quad0\quad1\quad3\quad3\quad4\quad...\\
		& 0\quad0\quad0\quad1\quad2\quad3\quad4\quad...k-1\quad k\quad k \quad k &
			(\text{\footnotesize Die Kurve beginnt dann mit $P_9$ und hört in $P_{n-1}$ auf})\\
		& 0\quad0\quad0\quad1\quad1\quad1 &(\text{\footnotesize $n=4$, Bézier-Splines})
\end{align*}
\begin{center}
 % 22.10
\end{center}
Für lauter verschiedene Knotenpunkte ist ein B-Spline der Ordnung $d$ $(d-1)$-mal stetig differenzierbar (von der
Klasse $C^d-1$)\\[1em]
Durch Wiederholen der gleihen Werte in der Knotenfolge kann man die Kurve näher an das Kontrollpolygon zwingen,
es geht aber Glattheit verloren!!
\begin{itemize}
 \item 2 gleiche Werte: $3 \quad 4 \quad 5 \quad 5 \quad 6 \quad 7 \quad ...$ Kurve berührt eine \emph{Strecke} des
	Kontrollpolygons, nur mehr $C^1$-stetig
	\begin{center}
	 % 22.11
	\end{center}
 \item 3 gleiche Werte: Kurve geht ddurch einen Kontrollpunkt, nur mehr $C^0$-stetig (stetig)
 \item 4 oder mehr gleiche Werte $\rightarrow$ unstetig
\end{itemize}