\subsection{Zusammenfassung}
\begin{center}
 % 18.1
\end{center}
Ein Lichtstrahl trifft auf eine Fläche
\begin{itemize}
 \item ein Teil wird diffus reflektiert
 \item ein Teil wird glänzend reflektiert
 \item ein Teil wird gespiegelt
 \item ein Teil wird (bei einem transparentem Stoff) durchgelassen und dabei gebeugt
 \item der Rest wird absorbiert.
\end{itemize}

\subsection{Einheiten}
In welchen Einheiten wird ein "`Lichtstrahl"' gemessen? (Die bisherige "`Intensität"' ist keine physikalische Einheit.)
\begin{description}
 \item[Lichtstrom] Energie an sichtbarem Licht pro Zeit
	\begin{center}
	 Einheit = 1 Lumen = 1 lm
	\end{center}
	Die analoge Einheit bei der Strahllungsmessung = 1 W, Lumen ist eingeschränkt auf sichtbares Licht, bezogen auf
	die Helligkeitsempfindlichkeit des Auges, abhängig von der Wellenlänge.
 \item[Lichtstärke] Wieviel Licht strahlt eine punktförmige Lichtquelle in eine bestimmte Richtung aus?
	\[\text{Lichtstärke} = \frac{\text{Lichtstrom}}{\text{Raumwinkel}}\]
	\begin{center}
	 Einheit = 1 Candela = 1 cd
	\end{center}
	Eine Lichtquelle, die mit 1 cd in alle Richtungen strahlt, hat einen Lichtstrom von $4 \pi$ lm.
	\begin{center}
	 % 12.2
	\end{center}
	[Raumwinkel: Der Raumwinkel, unter dem man ein Objekt sieht, \emph{ist} die Fläche der Projektion auf die
		Einheitskugel (analog zum Ebenenwinkel)]
	\begin{center}
	 % 12.3
	\end{center}
 \item[Beleuchtungsstärke] Auf eine Fläche einfallender Lichtstrom, pro Fläche
	\begin{center}
	 Einheit = 1 Lux = 1 lx = 1 $\dfrac{\text{lm}}{\text{m}^2}$\\[1em]
	 % 12.4
	\end{center}
	Bei der Planung von Beleuchtungsanlagen wichtig
 \item[Leuchtdichte] (engl. luminance) Wieviel Licht wird von einer ausgedehnten Lichtquelle pro Flächeneinheit
	in eine bestimmte Richtung abgestrahlt
	\begin{center}
	 $\dfrac{\text{Lichtstärke}}{\text{Fläche}} = \dfrac{\text{Lichtstrom}}{\text{Raumwinkel} \cdot \text{Fläche}}
	 = \dfrac{\text{Beleuchtungsstärke}}{\text{Raumwinkel}}$\\[1em]
	 % 12.5
	 Bild\\[1em]
	 Einheit = 1 $\dfrac{\text{cd}}{\text{m}^2}$
	\end{center}
	\begin{itemize}
	\item Die Leuchdichte ist für den visuellen Helligkeitseindruck maßgeblich.
	\item Die Leuchtdichte bleibt entlang eines Sichtstrahls konstant (sofern das Licht nicht durch Rauch, etc.
		abgeschwächt wird.)
	\item Bei den elektromagnetischen Strahlen ist die äquivalente Einheit die Strahlungsdichte
	\end{itemize}
	\begin{center}
	 % 12.6
	\end{center}
\end{description}

\subsection{Diffuse Reflexion}
Bei der diffusen Reflexion gilt das \textsc{Lambert}'sche Gesetz:
\begin{center}
 % 12.7
\end{center}
Die Leuchtdichte, die in eine bestimmte Richtung geschickt wird, die einen Winkel $\alpha$ mit der Normalen bildet
ist proportional zu $\boldsymbol{\cos \alpha}$.
\begin{center}
 % 12.8
\end{center}
Bei der Betrachtung von der Seite wird der gleiche Raumwinkel aus der Sicht des Betrachters von einer größeren Fläche
überdeckt (s. dazu auch Abschnitt \ref{sec:diffuse_refl}, S. \pageref{lambert}).
\begin{center}
 % 12.9
\end{center}
\begin{align*}
 \alpha = 0^\circ \qquad & 2 \tan \beta\\
	& \cos \alpha (\tan (\alpha + \beta) - \tan(\alpha - \beta))\\
 \lim_{\beta \to 0} \frac{\cos \alpha \tan (\alpha + \beta) - \tan(\alpha - \beta))}{2 \tan \beta}
	&= \cos \alpha  \lim_{\beta \to 0} \frac{\frac{1}{\cos (\alpha + \beta)} - \frac{-1}{\cos (\alpha - \beta)}}
		{\frac{2}{\cos^2 \beta}}\\
	&= \cos \alpha \frac{\frac{2}{\cos (\alpha}}
		{\frac{2}{\cos^2 \beta}} = \cos \alpha	
\end{align*}
Dieser effekt gleicht sich mit dem Lambert'schen Gesetz aus. Daher erscheint eine diffuse Fläche aus allen Richtungen
\emph{gleich hell}.\\
\section{Einsatz der physikalischen Grundlagen beim Raytracing}
Beim Raytracing wird der gespieglte Strahl (oder gebrochene Strahl) zurückverfolgt,
bis er auf eine diffuse Fläche trifft.
\begin{center}
 %12.10
\end{center}
Rekursives Raytracing an jedem Punkt, wo der Stahl auftrifft, werden die verschiedenen Anteile (diffus, gespiegelt,
gebrochen) aufgesammelt.
\begin{lstlisting}[morekeywords=function]
function trace(Strahl $r$; Faktor $f$; Schnitttiefe $t$)
	// Strahl = 1 Punkt und eine Richtung $\psline[linestyle=dotted](3,-0.5)(5,0.5)\psline{*->}(3,-0.5)(4,0)$
	berechne den Punkt $P$, wo der Stahl das Objekt trifft.
	if $P = \emptyset$ return (Hintergrundhelligkeit)
	$f := f \cdot e^{-l \cdot D}$	// $l$ = Laenge der Strecke bis zu P
					// $D$ = Daempfungsfaktor (fuer Nebel / Rauch)
					// $D = 0$ ... klare Luft, Dser Schritt ist optional
	$sum := 0$
	fuer jede Lichtquelle $Q$:
		Betrachte die Strecke $PQ$
		Wenn frei von Hindernissen;
		berechne Helligkeit nach dem Phong-Modell
		$sum := sum + \mathrm{Helligkeit} \cdot f$
	spiegele den Strahl an der Oberflaeche $\rightarrow$ Strahl $r'$;
	$sum := sum$ + trace($r'$, $\cdot f \cdot
			\underbrace{\mathrm{Spiegelungskoeffizient}}
			_{\footnotemark[1]}$, $t + 1$);
	falls transparent:
		brich den Strahl an der Flaeche $\rightarrow$ Strahl $r$
		$sum := sum$ + trace($r'$, $\underbrace{\mathrm{Brechungskoeffizient}}_{\footnotemark[2]} \cdot f$, t+1);
	return $sum$
\end{lstlisting}
\footnotetext[1]{hängt von der Geometrie und der Oberflächenbeschaffenheit von $P$ ab}
\footnotetext[2]{Welcher Antei des Lichts, das aus Richtung $r'$ kommt wird in Richtung $r$ weitergeleitet}

% Weiter s. Fotos 1-3