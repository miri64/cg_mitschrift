\section{Antialiasing}\label{sec:antialiasing}
verschiedene Ansätze:
\begin{enumerate}
 \item Betrachte den Bildpunkt als quadratische Fläche und nicht als Punkt, betrachte eine Kurve (1-dimensional) als Fläche (2-dimensional). z. B. Strecke als Rechtece
	\begin{center}
	 \psset{unit=0.5cm}
	 \begin{pspicture}(0,0)(5,5)
	  \psframe*(0,0)(5,5)
	  \rput{45}(2.5,2.5){
		\psframe*[linecolor=white](-2.5,-1.05)(2.5,-0.05)
	  }
	  \rput[r](-0.5,3){\rnode{49_l}{49\% weiß}}
	  \rput[l](5.5,2){\rnode{12_l}{12\% weiß}}
	  \psframe[fillstyle=hlines,hatchangle=60,linestyle=none,hatchsep=2pt](2,2)(3,3)
	  \pnode(2.5,2.4){49}
	  \psframe[fillstyle=hlines,hatchangle=60,linestyle=none,hatchsep=2pt](4,2)(5,3)
	  \pnode(4.2,2.8){12}
	  \psgrid[gridcolor=gray,gridlabels=0pt]
	  \ncline[linecolor=gray]{->}{49_l}{49}
	  \ncline[linecolor=gray]{->}{12_l}{12}
	 \end{pspicture}
	\end{center}
	Pixel wird entsprechend hell- oder dunkelgrau gefärbt. (im Allgemeinen als proportionale Mischung aller Farben,
	die im Pixelquadrat vorkommen).
	\begin{center}
		\psset{unit=0.5cm}
		\begin{pspicture}(0,0)(5,5)
			\rput{45}(2.5,2.5){
				\psframe*(-2.5,0)(2.5,1)
			}
			\rput{-45}(2.5,2.5){
				\psframe*[linecolor=red](-2.5,0)(2.5,-1)
			}
			\psframe[fillstyle=hlines,hatchangle=60,linestyle=none,hatchcolor=gray,hatchsep=2pt](2,2)(3,3)
			\psgrid[gridcolor=gray,gridlabels=0pt]
		\end{pspicture}
	\end{center}
	Aufwendig zu rechnen.
 \item Supersampling: Man rechnet mit einer höheren Pixeldichte als tatsächlich vorhanden.
	\begin{center}
		\psset{unit=0.75cm}
		\begin{pspicture}(0,0)(5,4)
			\rput{30}(2.2,2.5){
				\psframe*(-2.5,-1.1)(2.5,-0.1)
			}
			\psgrid[gridcolor=comment,gridlabels=0pt]
			\multirput(0,0)(0,1){4}{
				\multirput(0,0)(1,0){5}{
					\multirput(0.167,0.167)(0,0.333){3}{
						\multirput(0,0)(0.333,0){3}{
							\psdot[dotstyle=x,linecolor=gray]
						}
					}
				}
			}
			\pnode(3.5,1.5){1}
			\pnode(2.5,2.5){5}
			\rput[r](-0.5,3){\rnode{5l}{$\dfrac{5}{9}$ weiß}}
			\rput[l](5.5,2){\rnode{1l}{$\dfrac{1}{9}$ weiß}}
			\ncline[linecolor=gray]{->}{1l}{1}
			\ncline[linecolor=gray]{->}{5l}{5}
		\end{pspicture}
	\end{center}
	die verfeinerten Pixel werden "`binär"' zugeordnet (in Fläche und außerhalb) und entsprechend gesetzt.
	Die Werte der tatsächlichen Pixel ergeben sich als Mittelwert der in ihnen enthaltenen verfeinerten Pixel.
 \item Glättung:
	\paragraph*{Idee}
	\begin{center}
	 \begin{pspicture}(0,0)(10,1.4)
		\psset{linewidth=0.1mm}
		\multirput(0,1)(0.197,-0.034){10}{
			\pscircle(0,0){0.41}
			\pscircle(0,0){0.25}
			\pscircle(0,0){0.17}
			\pscircle(0,0){0.13}
			\pscircle(0,0){0.11}
			\psdot[dotsize=0.2cm](0,0)
		}
	 \end{pspicture}
	\end{center}
	Helligkeitswerte "`strahlen"' auf die Nachbarn ab.
	\begin{enumerate}
	\item berechne die Pixelweite zunächst ohne Antialiasing.
	\item Verteile die Helligkeit von jedem Pixel auf seinen Nachbarn nach einem festen Schema.
	\end{enumerate}
	\begin{center}
		\psset{unit=0.6cm,gridlabels=0pt}
		\begin{pspicture}(0,0)(3,3)
			\rput[r](-2,3){z. B.}
			\psgrid
			\rput(0.5,2.5){$\tfrac{1}{36}$}\rput(1.5,2.5){$\tfrac{4}{36}$}\rput(2.5,2.5){$\tfrac{1}{36}$}
			\rput(0.5,1.5){$\tfrac{4}{36}$}\rput(1.5,1.5){$\tfrac{16}{36}$}\rput(2.5,1.5){$\tfrac{4}{36}$}
			\rput(0.5,0.5){$\tfrac{1}{36}$}\rput(1.5,0.5){$\tfrac{4}{36}$}\rput(2.5,0.5){$\tfrac{1}{36}$}
		\end{pspicture}
	\end{center}
\end{enumerate}
3. lässt sich auch mit 2. kombinieren

\chapter{Helligkeit und Farbe in der Computergrafik}
\section{Helligkeit}
\Defi Helligkeit in der schwarz/weiß-Grafik bezeichnet einen Grauwert auf der Skala zwischen schwarz und weiß
	\begin{align}
	 \text{schwarz } &= 0, & \text{weiß } &= 1\\
	 \text{oder schwarz } &= 0, & \text{weiß } &= 25 \qquad \text{(8-Bit-Integer)}
	\end{align}
\paragraph*{Problem} tatsächliche Mischung 50\% weiß und 50\% schwarz $\rightarrow$ sehr hellesgrau
\paragraph*{Weber-Fechner'sches Gesetz} beschreibt die Nichtlinearität der Sinneswahrnehmungen (für Optik und
	Akustik gleichermaßen). Vergrößerung der Energie um einen \emph{konstanten Faktor} wird als Vergrößerung des Reizes in \emph{konstanten Schritten} wahrgenommen.
\Bsp	Eine Vergrößerung von 10 auf 12 Energieeinheiten wird genauso groß wahrgenommen wie eine Vergrößerung von
	5000 auf 6000.
\Bsp	Lautstärke wir in deziBel (dB) gemessen: Logarithmus aus der Schallenergie (oder Druck?). (Logarithmus aus
	konstanten Faktoren konstante Differenzen).
\Bsp	Eine Oktave in der Musik (z. B. Abstand zwischen tiefen C und hohen C) entspricht einer Verdoppelung der
	Schallfrequenz.
\begin{center}
	\begin{pspicture}(0,0)(5,3)
	 \psaxes{->}(0,0)(0,0)(5,3)
	 \psplot{0}{5}{2.7*x^3/125}
	 \rput(0,3.2){Intensität der Lichtbestrahlung}
	 \rput[l](5.2,0){Grauwert des Pixels}
	 \pstextpath[l](0,0){\psplot[linestyle=none]{1}{5}{2.7*x^3/125+0.1}}{Funktion der Form $y = x^\gamma$}
	\end{pspicture}
	% 8.7 
\end{center}
"`$\gamma$-Korrektur"', wird vom Bildschirm bzw. von der Grafikkarte automatisch durchgeführt.

\section{Additive Farbsysteme}
\subsection{Das RGB-Farbsystem}
In der Computergrafik geht man von einem 3-Komponenten-Modell aus: Farbe ist aus 3 Grundfarben zusammengemischt
\begin{center}
 {\color{red}rot (R)}, {\color{green}grün (G)}, {\color{blue}blau (B)}
\end{center}
(In Wirklichkeit: unendlich viele Grundfarben, für jede sichtbare Wellenlänge eine)
\begin{center}
 \psset{unit=0.25cm}
 \begin{pspicture}(0,0)(2.5,0.5)
  \multirput(0.25,0)(1.5,0){2}{
	\SpecialCoor
	\psdot[linecolor=red](0.36;-150)
	\psdot[linecolor=green](0.36;-30)
	\psdot[linecolor=blue](0.36;90)
	\pspolygon[linewidth=0.1mm](-0.75,-0.433)(0,0.866)(0.75,-0.433)
  }
  \multirput{180}(1,0.433)(1.5,0){2}{
	\SpecialCoor
	\psdot[linecolor=red](0.36;-150)
	\psdot[linecolor=green](0.36;-30)
	\psdot[linecolor=blue](0.36;90)
	\pspolygon[linewidth=0.1mm](-0.75,-0.433)(0,0.866)(0.75,-0.433)
  }
 \end{pspicture}
\end{center}
Auf einem Bildschirm sind Lichtpunkte (Phosphore) in drei Farben R, G, B nahe aneinander gitterförmig angeordnet,
Die Bildpunkte werden unabhängig von einander angesteuert.\\
Eine Farbe in der Computergrafik ist durch 3 RGB-Werte zwischen 0 und 1 (0,1, ..., 255) charakterisiert. 24 bit pro Pixel, $2^24 \approx 16$ Millionen Farben
\begin{center}
 \begin{tabular}{ll}
  \textbf{Farbe}	& $\boldsymbol{(r,g,b)}$\\
  rot			& (1,0,0)\\
  grün			& (0,1,0)\\
  blau			& (0,0,1)\\
  gelb (Y)		& (1,1,0)\\
  magenta (M)		& (1,0,1)\\
  cyan (C)		& (0,1,1)\\
  schwarz (K)		& (0,0,0)\\
  weiß			& (1,1,1)\\
  Grauwerte		& $(x,x,x)$ alle drei Werte sind gleich
 \end{tabular}\\
 \psset{unit=4cm,xMin=0,xMax=1.5,yMin=0,yMax=1.5,zMin=0,zMax=1.5,Alpha=60,Beta=20,linecolor=yellow,nameX=$r$,nameY=$g$,nameZ=$b$}
 \begin{pspicture}(-1,-0.5)(1,1.5)
  \pstThreeDCoor[linecolor=black]
  \pstThreeDBox(0,0,0)(1,0,0)(0,1,0)(0,0,1)
  \pstThreeDLine(0,0,0)(1,1,1)
  \pstThreeDPut(1,1,1){weiß}
  \pstThreeDPut(0.5,0.5,0.5){\rput{50}(0,0){Grauachse}}
  \pstThreeDPut(0,0,0){K}
  \pstThreeDPut(1,0,0){R}
  \pstThreeDPut(1,1,0){Y}
  \pstThreeDPut(1,0,1){M}
  \pstThreeDPut(0,1,0){G}
  \pstThreeDPut(0,1,1){C}
  \pstThreeDPut(0,0,1){B}
 \end{pspicture}\\
 Farbwürfel $[0,1]^3$
\end{center}
\begin{itemize}
 \item Das RGB-System ist für die intuitive Behandlung von Farben nicht geeignet
 \item die Grauwerte bilden die Grauachse $K$-Weiß im Farbwürfel.
 \item die übrigen Ecken bilden das Farbsechseck RYGCBM. Auf diesem Sechseck liegen die "`reinsten"'/"`stärksten"'
	Farben, alle anderen Farben kann man durch beimischen von Grau konstruieren.
\end{itemize}