\section{Kreise}
\begin{center}
	\psset{unit=0.5cm}
	\begin{pspicture}(0,0)(10,10)
		\psgrid[gridlabels=0pt]
		\psline[linecolor=gray,linestyle=dashed](8.5,5.5)(8.5,9.5)
		\psline[linecolor=gray,linestyle=dashed](7.5,6)(7.5,9.5)
		\psline[linecolor=gray,linestyle=dashed](6.5,6)(6.5,10)
		\psline[linecolor=gray,linestyle=dashed](6,6.5)(10,6.5)
		\psline[linecolor=gray,linestyle=dashed](5.5,8.5)(9,8.5)
		\psline[linecolor=gray,linestyle=dashed](5.5,7.5)(9.5,7.5)
		\psdot[linecolor=red](6.5,9.5)
		\psdot[linecolor=red](7.5,8.5)
		\psdot[linecolor=red](8.5,7.5)
		\psdot[linecolor=red](9.5,6.5)
		\psframe*[linecolor=red](0,3)(1,4)
		\psframe*[linecolor=red](0,4)(1,5)
		\psframe*[linecolor=red](1,2)(2,3)
		\psframe*[linecolor=red](2,1)(3,2)
		\pscircle(5,5){4.5}
		\psdot(5,5)
	\end{pspicture}
\end{center}

\paragraph*{Annahme}
	\begin{itemize}
	 \item	Radius $r$ ist ganzzahlig
	 \item	Mittelpunkt $(c_x,c_y)$ ist ein Gitterpunkt
	\end{itemize}	
	Kreisgleichung:
	\[(x-c_x)^2 + (y - c_y)^2 = r^2\]
	Betrachte den Fall, wo der Mittelpunkt (0,0) ist (anschließend alles um $(c_x,c_y)$ verschieben).
	Wir zeichnen den Bereich $x \ge 0, y \ge 0, y \ge x$ auf diesem Achtelkreis zeichnen wir auf jeder senkrechten
	Gittergeraden \emph{einen} Punkt.
	\begin{center}
	\psset{unit=0.3cm}
	\begin{pspicture}(-6,-6)(6,6)
		\rput[r](-12,5.5){Ausnutzung der Symmetrie:}
		\psgrid[gridlabels=0pt]
		\pscircle(0,0){6}
		\psdot(0,0)
		\psline(0,0)(5,5)
		\psline(0,0)(0,6)
		\psdot[linecolor=red](0,6)
		\psdot[linecolor=red](1,6)
		\psdot[linecolor=red](2,6)
		\psdot[linecolor=red](3,5)
		\psdot[linecolor=red](4,4)
		\psarc[linecolor=blue]{->}(0,0){6}{80}{90}
		\psarc[linecolor=blue]{<-}(0,0){6}{45}{60}
	\end{pspicture}
	\end{center}
	$x=i$ ist fest, Kreis verläuft zwischen $(i,j)$ und $(i,j+1)$.
	\begin{center}
	 \begin{pspicture}(0,0)(4,4)
		\psline(2,0)(2,4)
		\psline(1,0)(3,0)
		\psline(1,1)(3,1)
		\psline(1,2)(3,2)\uput{3pt}[0](3,2){$j$}
		\psline(1,3)(3,3)\uput{3pt}[0](3,3){$j+1$}
		\psline(1,4)(3,4)
		\psarc(0,-5){8}{60}{90}
	\end{pspicture}
	\end{center}
	Welchgen dieser beiden Punkte soll man auswählen?
	\begin{enumerate}
	 \renewcommand*\theenumi{(\arabic{enumi})}
	 \item	Wähle den Punkt, der kleineren \emph{Abstand} vom Kreis hat
	 \item	Berechne den Schnittpunkt mit der Geraden $x=i$ und $r$ und runde zum nächsten Gitterpunkt.
		\begin{itemize}
		 \item (Äquivalent: vergleiche den \emph{senkrechten Abstand} zum Kreis)
		 \item (Äquivalent: Lisegt $(i, j+\frac{1}{2})$ über oder under dem Kreisbogen?)
		\end{itemize}
	 \item	Wähle den Punkt der die \emph{Kreisgleichung} $x^2 + y^2 = r^2$ am besten erfüllt.
		\[|x^2 + y^2 - r^2| \to \mathrm{MIN}\]
	\end{enumerate}
	Für Geraden sind alle drei Bedingungen äquivalent\\
	Für (3). gibt es beliebig viele Varianten:
	\[\sqrt{x^2 + y^2} = r \Rightarrow |r - \sqrt{x^2 + y^2}| \to \mathrm{MIN} \text{ führt auf (1)}\]
	\[(x^2 + y^2)^2 = r^4 \Rightarrow |r^4 - (x^2 + y^2)^2| \to \mathrm{MIN} \text{ führt auf (1)}\]
	\begin{center}
	\begin{pspicture}(0,0)(4,4)
		\psdot(2,2.8)\pnode(2,2.8){i}
		\psline(0,1.5)(4,1.5)\psdot(2,1.5)\pnode(2,1.5){m}
		\psdot(2,0.2)\pnode(2,0.2){j}
		\psarc(0,0){3}{0}{90}
		\psline{<->}(2,0.2)(2.97,0.2)
		\nccurve[angleA=-60,angleB=60]{->}{i}{m}\naput{(2).}
		\nccurve[angleA=-120,angleB=120]{->}{m}{j}\nbput{(1).}
	\end{pspicture}
	\end{center}
\Satz	Wenn der Mittelpunkt $(c_x,c_y)$ und der Radius $r$ ganzzahlig sind, dann sind Bedingung (1),(2) und (3)
	äquivalent.\\
	\hrulefill
	Algebraische Formulierung von (1), (2), (3):
	\begin{itemize}
	 \item Punkt $(i,j+1)$ liegt außerhalb
		\[i^2 + (j + 1)^2 \ge r^2\]
	 \item Punkt $(i,j)$ liegt innerhalb
		\[i^2 + (j + 1)^2 < r^2\]
	\end{itemize}
	\newcommand{\glee}{\overset{>}{\underset{<}{=}}}
	\begin{enumerate}
	 \renewcommand*\theenumi{(\arabic{enumi})}
	 \item	$r-\sqrt{i^2+j^2} \glee \sqrt{i^2+(j+1)^2} - r$
	 \item	$i^2 + \left(j+\frac{1}{2}\right)^2 \glee r^2$
	 \item	$r^2 - (i^2 + j^2) \glee i^2+(j+1)^2 - r^2$
	\end{enumerate}
	\begin{align*}
	 (3) &\Longleftrightarrow 	& 2r^2	&\glee i^2 + j^2 + 2j + 1 + i^2 + j^2\\
		&		&	&= 2i^2 + 2j^2 + 2j + 1\\
	&\Longleftrightarrow	& i^2 + j^2 + j - r^2 + \frac{1}{2} &\glee 0 &&\Rightarrow\
		\text{"`="' kommt nicht vor für $i,j \in \mathbb{Z}$}\\
	 (2) &\Longleftrightarrow	& i^2 + j^2 + j + \frac{1}{4} - r^2 &\glee 0
	\end{align*}
	Für $i, j, r \in \mathbb{Z}$ sind (2) und (3) äquivalent:
	\begin{empheq}[box=\fbox]{align*}
		\overbrace{i^2 + j^2 + j - r^2}^{g(i,j+1),\ \text{s. Algorithmus}} &\ge 0 &&\Rightarrow\ \text{Zeichne $(i,j)$}\\
		 &\le -1 &&\Rightarrow\ \text{Zeichne $(i,j+1)$}
	\end{empheq}

	\paragraph*{Behauptung}
	\begin{enumerate}
	 \renewcommand*\theenumi{(\alph{enumi})}
	 \item $i^2 + \left(j + \frac{1}{2}\right)^2 \ge r^2 \Rightarrow \sqrt{i^2+(j+1)^2} - r > r - \sqrt{i^2+j^2}$
	 \item $i^2 + j^2 + j + \frac{1}{2} \le r^2 \Rightarrow \sqrt{i^2+(j+1)^2} - r < r - \sqrt{i^2+j^2}$
	\end{enumerate}
	Daraus folgt, dass Regel (1) mit den beiden anderen Regeln (2), (3) konsistent ist.\pagebreak
\Bew	(von (b).) Die Funktion $f(t) = \sqrt{t}$ ist fü $t > 0$ konkav
	\begin{center}
	\begin{pspicture}(0,-1)(4,4)
		\psarc(4,0){4}{90}{180}
		\psline{->}(0.5,0)(4,0)
		\psline(1,-0.1)(1,0.1)\uput{1pt}[-90](1,-0.1){$u$}
		\psline(2,-0.1)(2,0.1)\uput{1pt}[-90](2,-0.1){$\dfrac{u+v}{2}$}
		\psline(3,-0.1)(3,0.1)\uput{1pt}[-90](3,-0.1){$v$}
		\psline(1,0)(1,2.64)(3,3.87)(3,0)
		\psdot(1,2.64)\uput{1pt}[138.65](1,2.64){$f(u)$}
		\psdot(2,3.255)\rput[tl](4.5,2){$\dfrac{f(u)+f(v)}{2}$}
		\psline{->}(4.5,2)(2,3.255)
		\psdot(3,3.87)\uput{1pt}[104.49](3,3.87){$f(v)$}
	\end{pspicture}\\
	 (daraus folgt: $f\left(\dfrac{u+v}{2}\right) \ge \dfrac{f(u)+f(v)}{2}$)
	\end{center}
	Eine differenzierbare Funktion $f$ ist konkav $\Leftrightarrow$ $f'$ monoton fallen $\Leftrightarrow$ $f'' \le 0$
	\[f'(t) = \frac{1}{2 \sqrt{t}} \searrow\ \text{konkav}\]
	\begin{align*}
	 u &= i^2 + j^2\\
	 v &= i^2 + (j+1)^2 = i^2 + j^2 + 2j + 1\\
	 \frac{u+v}{2} &= i^2 + j^2 + j + \frac{1}{2}\\
	 r &\underset{\mathrm{N.V.}}{\ge} \sqrt{i^2 + j^2 + j + \frac{1}{2}} \ge \frac{1}{2} \left[\sqrt{i^2} + \sqrt{i^2+(j+1)^2}\right]\\
	 2r & \ge \sqrt{i^2+j^2} + \sqrt{i^2 + (j+1)^2}
	\end{align*}
\Bew	(von (a).) Funktion $h(y) = \sqrt{i^2 + y^2}$ ist \emph{konvex}
	\begin{align*}
	 h'(y) &= \frac{1 \cdot \cancel{2} y}{\cancel{2}\sqrt{i^2+y^2}} = \frac{1}{\sqrt{\frac{i^2}{y^2}+1}}
			\nearrow\text{ monoton wachsend}\\
	 g\left(\frac{u+v}{2}\right) &\le \frac{1}{2} (g(u) + g(v)) \qquad u = j, v = j+1\\
	 r^2 &\underset{\mathrm{N.V.}}{\le} \sqrt{i^2+\left(j+\frac{1}{2}\right)^2}
		\le \frac{1}{2}\left(\sqrt{i^2+j^2}+\sqrt{i^2 + (j+1)^2}\right)
	\end{align*}
\paragraph*{Algorithmus}
	beginnt mit $(0,r)$ und zeichnet Punkte von links nach rechts.
	\begin{itemize}
	 \item letzter gezeichneter Punkt $= (i-1,j)$
	 \item soll nächster $(i,j)$ oder $(i,j-1)$ gezeichnet werden?
		\begin{align*}
			g(i,j) &= i^2 + (j-1)^2 + (j-1) - r^2 = i^2 + j^2 - 2j \cancel{+ 1} + j \cancel{- 1} r^2\\
			g(i,j) &:= i^2 + j^2 - j - r^2 \begin{cases}
			                              \ge 0 &\Rightarrow\ \text{zeichne $(i, j-1)$}\\
			                              \le -1 & \Rightarrow\ \text{zeichne $(i, j)$}
			                             \end{cases}
		\end{align*}
		\begin{align*}
		 g(i+1,j) - g(i,j) &= (i+1)^2 - i^2 = 2i + 1\\
		 g(i,j-1) - g(i,j) &= (j-1)^2 - (j-1) - j^2 + j\\
				   &= j^2 - 2j + 1 - j + 1 - j^2 + j = -2j + 2
		\end{align*}
		\begin{center}
			\vspace{0.5cm}
			\begin{pspicture}(0,3)(3,6)
				\psgrid[gridlabels=0pt]
				\psline(0,3)(0,6)
				\rput[b](0,6){$(0,r)$}
				\rput[lb](2,5.5){$g(0,r) = -r$}
				\psline{->}(0,6)(1,6)
				\psline{->}(1,6)(1,5)
				\psline{->}(1,5)(2,5)
				\psline{->}(2,5)(2,4)
				\psdot[linecolor=red](0,6)
				\psdot[linecolor=red](1,6)
				\psdot[linecolor=red](1,5)
				\psdot[linecolor=red](2,5)
			\end{pspicture}
		\end{center}
	\end{itemize}
	\begin{lstlisting}[language=,keywords={loop,if,then, until},mathescape=true]
i := 0; j := r, g := -r; // Invariante: g = g(i,j)

loop
	SetPixel(i,j)
	g := g + 2i + 1; i := i + 1
	if g $\ge$ 0 then j := j - 1; g := g - 2j 
until j < i
	\end{lstlisting}

	\begin{lstlisting}[language=,mathescape=true]
SetPixel(c$_x$+i,$c_y$+j)
SetPixel(c$_x$+j,$c_y$+i)
SetPixel(c$_x$-i,$c_y$+j)
SetPixel(c$_x$-j,$c_y$+i)
SetPixel(c$_x$-i,$c_y$-j)
SetPixel(c$_x$-j,$c_y$-i)
SetPixel(c$_x$+i,$c_y$-j)
SetPixel(c$_x$+j,$c_y$-i)
	\end{lstlisting}
	Die Pixel in de n 4 Himmelsrichtungen werden doppelt gezeichnet.

\section{Schwachstellen der Rasterung (Aliasing)}
\begin{itemize}
 \item merkbare Sprünge bei fast achsenparallelen Geraden
	\begin{center}
	 \psset{unit=0.2cm}
	 \begin{pspicture}(0,0)(22,4)
	  \psgrid[gridlabels=0pt]
	  \psframe*(-2,1)(11,2)
	  \psframe*(11,2)(25,3)
	  \psline{->}(14,0)(11,2)
	  \rput[lt](14,0){deutlich sichtbarer Knick}
	 \end{pspicture}
	\end{center}
 \item unterschiedliche Helligkeitsverteilung in Abhängigkeit von der Steigung
	\begin{center}
	 \psset{unit=0.2cm}
	 \begin{pspicture}(0,0)(22,15)
	  \psgrid[gridlabels=0pt]
	  \psframe*(-2,13)(24,14)
	  \multirput(12,-1)(1,1){12}{
		\psframe*(0,0)(1,1)
	  }
	 \end{pspicture}
	\end{center}
	\begin{itemize}
	\item	horizontale/vertikale Linie hat 1 Pixel pro Längeneinheit
	\item	schräge Linie $(45^\circ)$ hat 1 Pixel pro $\sqrt{2}$ Längeneinheiten
	\end{itemize}
	$\Rightarrow$ schräge Linien erscheinen dünner
	\item	Buchstaben könne verschieden breit werden
	\begin{center}
		\psset{unit=0.5cm}
		\begin{pspicture}(0,0)(2,7)
			\rput[b](1,0){
				\includegraphics[width=1cm,height=3.25cm]{bigi.eps}
			}
			\psgrid[gridlabels=0pt]
			\psdot(0,0)\psdot(0,4)
			\psdot(1,0)\psdot(1,1)\psdot(1,2)\psdot(1,3)\psdot(1,4)\psdot(1,6)
			\psdot(2,0)
		\end{pspicture}
		\hspace{2cm}
		\begin{pspicture}(0,0)(3,7)
			\rput[b](1.5,0){
				\includegraphics[width=1cm,height=3.25cm]{bigi.eps}
			}
			\psgrid[gridlabels=0pt]
			\psdot(0,0)\psdot(0,4)
			\psdot(1,0)\psdot(1,1)\psdot(1,2)\psdot(1,3)\psdot(1,4)\psdot(1,6)
			\psdot(2,0)\psdot(2,1)\psdot(2,2)\psdot(2,3)\psdot(2,4)\psdot(2,6)
			\psdot(3,0)
		\end{pspicture}
	\end{center}

\end{itemize}
\paragraph*{Lösung (Antialiasing)}
	Verschiedene Graustufen für Pixel, statt nur schwarz und weiß


